\documentclass{article}
\usepackage[utf8]{inputenc}
\usepackage[a4paper, total={6in, 8in}]{geometry}
\usepackage{graphicx}
\usepackage{amsmath,amsfonts}
\usepackage{gensymb}


\title{15-411: L6 Written Report}
\author{Manganese}
\date{December 14, 2015}

\begin{document}

\maketitle

\section{Introduction}

Blah blah WE NEED TO DO THIS\\
\section{Language Extensions}

Since type inference will cause some tests to typecheck that previously would have thrown an error, it is disabled by default, and can be enabled with the compilation flag --typeInference (i.e. ./top.native --typeInference $<$filename$>$).\\

\subsection{Strings and characters}
String constants and characters are now included in the $exp$ type from the L4 grammar. String constants are defined as the following regexp: (``) + (any character that isn't double quotes $\cup \  \backslash")^*$ +("). Characters are defined as any OCaml character, plus several that don't appear in OCaml apparently, just as `$\backslash$ f'.

The semantics for strings and characters are just as defined in the handout: characters can be compared by their ascii values with $<. >, ==$, etc, strings cannot be compared but are treated as a small type.

\subsection{Function pointers}
The addition of function pointers to the L5 grammar was straightforward. A new type of gdecl (global declaration) was added with the form \begin{align*}
\textbf{typedef} \ \textrm{fdecl}
\end{align*}
The unary address-of operator $\&$ was also added, along with the new expression for a function pointer call.

The main interesting thing about the semantics for function pointers is the requirement that comparison is done nominally, as it tested by our test ``test18.l6".

\subsection{Basic type inference}
The addition of type inference to the L6 grammar was also straightforward. The main differences are that fdecls and fdefns not longer must begin with a type. So fdecls can also be of the form
\begin{align*}
\textbf{ident} \ \textrm{param-list};
\end{align*}
and fdefns can be also be of the form
\begin{align*}
\textbf{ident} \ \textrm{param-list} \ \textrm{block}
\end{align*}
Similarly, params no longer must begin a type. So params can also simply be an \textbf{ident}.

The semantics also required significant changes. On the more basic side, variables no longer must be declared before they are defined (although they must still be initialized). So something like\\
\\
main () \{\\
 x = 0;\\
 return x;\}\\
 \\
is ok now; the type of main is inferred to be int, and the type of x is inferred to be int. We also still only allow a variable to have single type in a given scope, so\\
\\
main () \{\\
 x= true;\\
 x = 0;\\
 return x;\}\\
 \\
is disallowed. 

Note that type annotations are still allowed of course. We also made \textbf{alpha} a keyword, and included a new type \textbf{alpha}$<n>$, where $n$ is any positive integer (so \textbf{alpha1}, \textbf{alpha2}, etc). The alpha type simply implies no type information: it is equivalent to leaving off the annotation.\\

There were other changes as well: BILLY DESCRIBE THESE

\subsection{Partial type annotations}
Also inspired by functional programming, we support ``partial type annotations". We define a partial type annotation to be an annotation which provides more information than just \textbf{alpha}, but does not fully specify the type. An example would be something like this:\\
\\
main () \{\\
alpha1[] a = alloc$\_$array(int[], 5);\\
return 0;\\
\}\\
\\
The type annotation on a means that a is ``some type of array". The type of the right-hand-side is int[][], which is indeed a type of array, so this typechecks.\\

\subsection{Polymorphism}
In order to make type inference more interesting, we decided to support some amount of polymorphism, specifically allowing polymorphic functions. For example, the function\\
\\
f(x) \{ return x; \}\\
\\
has type alpha1 $\to$ alpha1. Thus it can be called with any type as argument, so the following typechecks:\\
\\
main () \{\\
p = alloc(int);\\
return f(*p) + *f(p);\\
\}\\
\\
Unfortunately, there are many subtleties to the these semantics that we did not have time to
implement. For example, alloc(alpha) should be disallowed, since the size of alpha is unknown, but alloc(alpha []) should be allowed. We imagine there are quite a few other such edge cases.

We also decided to completely ignore type inference for structs, because for something like:\\
\\
struct s1 \{x;\}\\
\\
The size of x is needed at compile time. Even if you require type annotations on fields, we were not sure how to deal with the following:\\
\\
struct s1 \{int x; int y;\}\\
\\
struct s2 \{int y;\}\\
\\
f(s) \{return s$\to$y;\}\\
\\
Since it is now impossible to compute the field offset of y at compile time. There are even more issues with structs beyond these, but these were enough to deter us from even attempting to handle structs.\\
\\

\section{Compilation}

\section{Examples}

\section{Analysis}


\end{document}



